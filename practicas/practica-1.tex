%%%%%%%%%%%%%%%%%%%%%%%%%%%%%%%%%%%%%%%%%%%%%%%%
\documentclass[a4paper,oneside,fleqn,11pt]{article}
\usepackage[solutions]{practicas}
\usepackage{babel}

\overfullrule5pt

\Materia{Estadística e inferencia I}
\Cuatrimestre{Primer Cuatrimestre --- 2024}
\Practica{1}
\Titulo{Estimadores [\textcolor{red}{v0.1}]}

%%%%%%%%%%%%%%%%%%%%%%%%%%%%%%%%%%%%
% Abreviaturas
\newcommand\CC{\mathbb{C}}
\newcommand\RR{\mathbb{R}}
\newcommand\QQ{\mathbb{Q}}
\newcommand\ZZ{\mathbb{Z}}
\newcommand\NN{\mathbb{N}}
\newcommand\PP{\mathbb{P}}
\newcommand\N{\mathcal{N}}
\newcommand\abs[1]{\lvert#1\rvert}
\renewcommand\emptyset{\varnothing}

\DeclareMathOperator{\GL}{GL}

\newcommand\sg{\mathrm{sgn}}

\newcommand\Un{\mathcal{U}}

\newcommand\ord{\mathrm{ord}}

%%%%%%%%%%%%%%%%%%%%%%%%%%%%%%%%%%%%
% proclaiming things
\theoremseparator{.}
\newtheorem*{Definicion}{Definición}
\theoremindent\parindent
\newtheorem*{Teorema}{Teorema}
\newtheorem*{Proposicion}{Proposición}

%%%%%%%%%%%%%%%%%%%%%%%%%%%%%%%%%%%%%%%%%%%%%%%%%%%%%%%%%%%%%%%%%%%%%%%%%%%%%%
\begin{document}

\maketitle

\begin{ejercicios}

%%%%%%%%%%%%%%%%%%%%%%%%%%%%%%%%%%%%%%%
\item
Se analizó una muestra de 12 piezas de pan blanco de cierta marca y se
determinó el porcentaje de carbohidratos contenido en cada una de las piezas,
obteniéndose los siguientes valores:
\begin{align*}
  & 76.93, && 76.88, && 77.07, && 76.68, && 76.39, && 75.09, \\ 
  & 77.67, && 76.88, && 78.15, && 76.50, && 77.16, && 76.42.
\end{align*}

\begin{ejitems}
    \item Estimar el promedio del porcentaje de carbohidratos contenido en las
piezas de pan de esta marca.
    
    \item Estimar la mediana del porcentaje de carbohidratos.
    
    \item Estimar la proporción de piezas de pan de esta marca cuyo contenido
de carbohidratos no excede el 76.5\%.
\end{ejitems}

\item
Se supone que la longitud en milímetros de cierto tipo de eje tiene una
distribución normal con desvío estándar $\sigma = 0.05$. Se toma una muestra
de 20 ejes y se observa que la longitud media de los ejes es de $52.3$.
\begin{ejitems}
  \item Hallar un intervalo de confianza para la verdadera longitud media de
nivel 0.99.
    
  \item ¿Qué tamaño debe tener la muestra para que la longitud de un
intervalo de nivel 0.99 sea a lo sumo 0.03?
\end{ejitems}


\item
La nota de una prueba de aptitud sigue una distribución normal. Una muestra aleatoria de nueve alumnos de la ciudad arroja los siguientes resultados: $5, 8.1, 7.9, 3.3, 4.5, 6.2, 6.9, 7.5, 9.1$.

\begin{ejitems}
    \item Hallar un intervalo de confianza para la nota media de los alumnos de la ciudad.
    
    \item La nota media de todos los alumnos de la ciudad en ese mismo examen el año anterior es $7.50$. ¿Le parece que hay motivos para afirmar que la nota media de los alumnos ha cambiado con respecto al año anterior?
\end{ejitems}



\item
Considere muestras aleatorias de cada una de las siguientes distribuciones:
\begin{enumerate}[label=\roman*]
  \item normal de parámetros $\mu$ y $\sigma^2$;
  \item exponencial de parámetro $\lambda$;
  \item Poisson de parámetro $\lambda$;
  \item con PDF que depende de un parámetro $\theta\in (0,1)$; 
  \begin{align*}
    f(x;\theta) 
    = \frac{1}{\theta}x^{(\frac{1}{\theta}-1)}I_{[0,1]}(x), 
    \quad \text{donde  } 
    I_{[0,1]}(x) = 
    \begin{cases*} 
      1 & si $0 \leq x \leq 1$, \\ 
      0 & caso contrario;
    \end{cases*}
  \end{align*}
  \item geométrica de parámetro $p$;
  \item gamma de parámetros $\alpha$ y $\lambda$;
  \item uniforme $\mathcal{U}[0,\theta]$. 
\end{enumerate}
\begin{ejitems}
  \item Hallar en cada caso el estimador de momentos de los parámetros
  \item Hallar en cada caso salvo vi
el estimador de máxima verosimilitud de los parámetros (salvo vi)
  \item Para i,ii,iii y vii decir si los estimadores son insesgados o 
asintóticamente insesgados.
  \item Calcular el ECM de los estimadores de $\theta$ en vii. Comparando
uno con otro, ¿cuál de los dos estimadores usarías?
\end{ejitems}





\end{ejercicios}
\end{document}
%%%%%%%%%%%%%%%%%%%%%%%%%%%%%%%%%%%%%%%%%%%%%%%%%%%%%%%%%%%%
