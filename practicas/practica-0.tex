%%%%%%%%%%%%%%%%%%%%%%%%%%%%%%%%%%%%%%%%%%%%%%%%
\documentclass[a4paper,oneside,fleqn,11pt]{article}
\usepackage[solutions]{practicas}
\usepackage{babel}

\overfullrule5pt

\Materia{Estadística e inferencia I}
\Cuatrimestre{Primer Cuatrimestre --- 2024}
\Practica{0}
\Titulo{Repaso [\textcolor{red}{v0.1}]}

%%%%%%%%%%%%%%%%%%%%%%%%%%%%%%%%%%%%
% Abreviaturas
\newcommand\CC{\mathbb{C}}
\newcommand\RR{\mathbb{R}}
\newcommand\QQ{\mathbb{Q}}
\newcommand\ZZ{\mathbb{Z}}
\newcommand\NN{\mathbb{N}}
\newcommand\PP{\mathbb{P}}
\newcommand\N{\mathcal{N}}
\newcommand\abs[1]{\lvert#1\rvert}
\renewcommand\emptyset{\varnothing}

\DeclareMathOperator{\GL}{GL}

\newcommand\sg{\mathrm{sgn}}

\newcommand\Un{\mathcal{U}}

\newcommand\ord{\mathrm{ord}}

%%%%%%%%%%%%%%%%%%%%%%%%%%%%%%%%%%%%
% proclaiming things
\theoremseparator{.}
\newtheorem*{Definicion}{Definición}
\theoremindent\parindent
\newtheorem*{Teorema}{Teorema}
\newtheorem*{Proposicion}{Proposición}

%%%%%%%%%%%%%%%%%%%%%%%%%%%%%%%%%%%%%%%%%%%%%%%%%%%%%%%%%%%%%%%%%%%%%%%%%%%%%%
\begin{document}

\maketitle

\begin{ejercicios}

%%%%%%%%%%%%%%%%%%%%%%%%%%%%%%%%%%%%%%%
\section{Probabilidad}

\item 
Un estudio sobre la relación entre nivel de ingresos (A=alto, M=medio, B=bajo) y
la preferencia por una de las tres grandes marcas de automóviles (Y,W,Z) da como
resultado la siguiente tabla de probabilidades conjuntas.

\begin{table}[h]
    \centering
    \begin{tabular}{c|ccc|c}
         & B & M & A \\
        \hline
        Y & 0.10 & 0.13 & 0.02 & 0.25 \\
        W & 0.20 & 0.12 & 0.08 & 0.40 \\
        Z & 0.10 & 0.15 & 0.10 & 0.35 \\
        \hline
         & 0.40 & 0.40 & 0.20 \\
    \end{tabular}
\end{table}

\noindent 
Esta tabla muestra, por ejemplo, que $P(\text{ingreso bajo y preferencia } Y) =
P(B \cap Y) = 0.10$, que $P(\text{ingreso bajo}) = P(B) = 0.40$ y que
$P(\text{preferencia } Y) = P(Y) = 0.25$.

\begin{ejitems}
  \item
  Calcular las siguientes probabilidades condicionales:
  \begin{align*}
    & P(W|A)
    && P(M|Z)
    && P(Y^c|M)
    \\
    & P(M|Y^c)
    && P(M|W \cup Z)
    && P(B \cup M|Z)
  \end{align*}
  
  \item
  ¿Cuál es la probabilidad de que una persona elegida al azar prefiera la marca Y
  o tenga un alto ingreso?
\end{ejitems}

\item
Supongamos que se descubre una nueva enfermedad, fatal pero muy rara: la tiene
una de cada mil millones de personas. Para detectar esta enfermedad se
desarrolló un test que tiene una tasa de falsos positivos de 1 en un millón.
Es decir que si el resultado es positivo, la probabilidad de no tener esa
enfermedad es de $1$ en $1.000.000$.

Nuestro héroe se hace el test y le da... ¡positivo! ¡un test tan preciso! A no
desesperar:

\begin{ejitems}
  \item En el mundo hay aproximadamente 8 mil millones de personas: ¿cuántas
  personas aproximadamente sufren de esta enfermedad?  
  \item ¿Y aproximadamentea cuántas personas en el mundo les da positivo?  
  \item De los que salen positivos en el test, ¿cuántos tienen esta enfermedad?
  \item ¿Cuál es la probabilidad de que nuestro héroe tenga esa enfermedad rara?
  \item Formalizar las cuentas anteriores usando conceptos de probabilidad condicional y el Teorema de Bayes.
\end{ejitems}

\item Calcular la esperanza $E(X)$ y la varianza $V(X)$ cuando
\begin{ejitems}
  \item $X$ es una Bernoulli con parámetro $p\in(0,1)$.
  \item $X$ es una binomial con parámetros $p\in(0,1)$ y $n\in\NN$.
\end{ejitems}


\section{PDF y CDF}

Sea $X$ una variable aleatoria a valores reales. La \emph{función de
distribución acumulada}, o \emph{CDF} por sus siglas en inglés, es $F(x) =
P(X\leq x)$ para cada $x\in\RR$.  Si $X$ es continua, la \emph{función de
densidad de probabildad}, o \emph{PDF} por sus siglas en inglés, es la
derivada de $F$.

%%%%%%%%%%
\item
Sea $X$ una variable aleatoria con distribución $\N(5, 0.25)$. Calcular
\begin{ejitems}
    \item $P(4.75 \leq X \leq 5.50)$,
    \item $P(|X| > 5.25)$,
    \item el valor de $c$ tal que $P(|X - 5| \leq c) = 0.90$,
    \item el 90-percentil de $X$.
\end{ejitems}

\item
La porción de memoria ocupada en un servidor de un sistema de terminales en
red es una variable aleatoria continua $X$ que toma valores entre $0$ (sin
carga) y $1$ (carga completa). La PDF de $X$ está dada por 
\[
  f(x) = 
  \begin{cases}
    4x^3 & \text{si $0 < x < 1$;} \\
    0 & \text{en caso contrario.}
  \end{cases}
\]
\begin{ejitems}
  \item Hallar la mediana de la porción ocupada de memoria.
  
  \item Deducir la densidad de la variable que mide la porción de memoria que
  falta ocupar, es decir $Z = 1 - X$.
\end{ejitems}

\item
Una muestra de tabaco puede provenir de dos variedades distintas, I y II, con
probabilidades 0.35 y 0.65 respectivamente. El contenido de nicotina es una
variable aleatoria, cuya distribución es $\N(1.9, 0.16)$ en la variedad I y
$\N(2.2, 0.09)$ en la variedad II.

\begin{ejitems}
  \item 
  Hallar la probabilidad de que en una muestra elegida al azar el contenido de
  nicotina sea mayor o igual que $2.1$.
  
  \item
  Hallar la probabilidad de que dado que el contenido de nicotina es mayor que
  $2.1$, la muestra provenga de la variedad I.
\end{ejitems}

\item 
Para cada una de las siguientes distribuciones graficar la PDF y la CDF. 
Calcular media, mediana, moda, error estándar y, si amerita, algunos percentiles.
\begin{ejitems}
  \item Normal estándar $\N(0,1)$.
  \item Exponencial con parámetro $\lambda = 1/2$.
  \item Uniforme continua en el intervalo $(0, 1)$.
  \item Distribución $t$ de Student con 10 grados de libertad, y con 100.
\end{ejitems}

\section{Teorema central del límite}

  \item Para cada $n$ entre 1 y 3000, generar observaciones $x_1, \ldots, x_n$
  de $X_1, \ldots, X_n$ vaiid con distribución exponencial con parámetro
  $\lambda = 1/2$, con $\lambda$ a elección. Luego, obtener $\bar x_n$ como la
  media de estas observaciones, es decir, 
  \[
    \bar x_n = \frac{1}{n} \sum_{i=1}^{n} x_i.
  \]
  Graficar $n$ vs $x_n$ e interpretar.
  
  \item 
  Este ejercicio es una constatación experimental del Teorema Central del
  Límite. 
  \begin{ejitems}
  \item Considerar dos observaciones $x_1$ y $x_2$ de variables aleatorias $X_1$
  y $X_2$ independientes con distribución exponencial con parámetro
  $\lambda=1/2$ y guardar el promedio de ambas, es decir, $\bar x_2$. 
  Repetir este proceso 1000 veces y a partir de los valores obtenidos, realizar
  un histograma.
  
  \item Aumentar a diez las variables promediadas, es decir, considerar ahora $n
  = 10$ observaciones de variables aleatorias independientes con la misma
  distribución del ítem anterior. Repetir este proceso 1000 veces y a partir de
  los valores obtenidos, realizar un histograma.
  
  \item Hacer histogramas para $n = 100$ y $n=1000$. Encontrar parámetros $\mu$
  y $\sigma$ para que la CDF de una $\N(\mu,\sigma)$ se superponga lo mejor
  posible al histograma que corresponde a $n=1000$.
\end{ejitems}

\end{ejercicios}
\end{document}
%%%%%%%%%%%%%%%%%%%%%%%%%%%%%%%%%%%%%%%%%%%%%%%%%%%%%%%%%%%%
